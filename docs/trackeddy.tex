\documentclass[paper=a4, fontsize=12pt]{article}
\usepackage[english]{babel} 
\usepackage[utf8]{inputenc}
\usepackage{geometry}
\usepackage{xcolor}
\usepackage{listings}
\lstset{basicstyle=\ttfamily,
	showstringspaces=false,
	commentstyle=\color{red},
	keywordstyle=\color{blue}
}

\lstdefinestyle{bash}
{
	backgroundcolor=\color{gray},
	basicstyle=\small\color{white}\ttfamily
}

\geometry{
	a4paper,
	total={170mm,257mm},
	left=25mm,
	right=25mm,
	top=25mm,
	bottom=25mm
}
\usepackage[UKenglish]{isodate}
\title{TrackEddy}
\author{Josu\'e Mart\'inez Moreno }
\cleanlookdateon
\begin{document}
\maketitle
This source code will let you identify all the eddies in the ocean, but also It can be adapted for any other normal function in a 2D surface.
\section{Installation}
\subsection{Get Code}
\begin{enumerate}
\item Make a new directory where you want the repository.
\item Clone the TrackEddy repository from Github. In the command prompt, type:
\begin{lstlisting}[language=bash,style=bash]
 git clone https://github.com/Josue-Martinez-Moreno/trackeddy.git
\end{lstlisting}
\item Install the package globally: 
\begin{lstlisting}[language=bash,style=bash]
pip install -e .
\end{lstlisting}
This make the package an editable install so that it can be updated with future additions to TrackEddy. To instead install the package locally:
\begin{lstlisting}[language=bash,style=bash]
pip install --user -e .
\end{lstlisting}
\end{enumerate}
\subsection{Update the code later}
\begin{enumerate}
\item Move into your TrackEddy directory.
\item Update your GitHub repository. 
\begin{lstlisting}[language=bash,style=bash]
git pull
\end{lstlisting}
Edit your install of TrackEddy.
\begin{lstlisting}[language=bash,style=bash]
pip install -e .
\end{lstlisting}
or 
\begin{lstlisting}[language=bash,style=bash]
pip install --force-reinstall -e . 
\end{lstlisting}
or, for local installation: 
\begin{lstlisting}[language=bash,style=bash]
pip install --ignore-installed --user .
\end{lstlisting}
\end{enumerate}
\section{Test the code}
\textcolor{red}{Work in progress!}
\section{Usage}
\textcolor{red}{Work in progress!}
\section{Learn more about TrackEddy.}
\textcolor{red}{Work in progress!}
\end{document}